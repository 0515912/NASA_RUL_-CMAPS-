%TCIDATA{LaTeXparent=0,0,RUL1.tex}

\title{A Neural Network-Evolutionary Computational Framework for Remaining Useful Life Estimation of Mechanical Systems}

\author{David Laredo\thanks{David Laredo, Zhaoyin Chen and J. Q. Sun are with Department of Mechanical Engineering, University of California, Merced, CA 95343.}, Zhaoyin Chen, Oliver Sch\"utze\thanks{Oliver Sch\"utze is with Department of Computer Science, CINVESTAV, Mexico City, Mexico} and J. Q. Sun}%

\markboth{IEEE Transactions on Industrial Informatics}{Shell \MakeLowercase{\textit{et al.}}: Bare Demo of IEEEtran.cls for Journals}

\maketitle

\begin{abstract}
This paper presents a framework for estimating the remaining useful life (RUL) of mechanical systems. The framework consists of a multi-layer perceptron and an evolutionary algorithm for optimizing the data-related parameters. The framework makes use of a strided time window along with a piecewise linear model to estimate the RUL for each time window in the training set. Tuning the data-related parameters in the optimization framework allows for the use of simple models, e.g. neural networks with few hidden layers and few neurons at each layer, which may be deployed in environments with limited resources such as embedded systems. The proposed method is evaluated on the publicly available CMAPS dataset. The accuracy of the proposed method is compared against other state-of-the art methods in the literature. The proposed method is shown to perform better while making use of compact model.
\end{abstract}


\begin{keywords}
artificial neural networks, moving time window, RUL estimation, CMAPS, prognostics, evolutionary algorithms
\end{keywords}

%\IEEEpeerreviewmaketitle
