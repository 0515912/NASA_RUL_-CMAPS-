\section{Introduction}
\label{sec:rul_intro}

Traditionally, maintenance of mechanical systems has been carried out based on scheduling strategies, nevertheless such strategies are often costly and less capable of meeting the increasing demand of efficiency and reliability \citep{Gebraeel2005, Zaidan2013}. Condition based maintenance (\gls{cbm}) also known as intelligent prognostics and health management (\gls{phm}) allows for maintenance based on the current health of the system, thus cutting costs and increasing the reliability of the system \citep{Zhao2017}. To avoid confusion, here we define prognostics as the estimation of remaining useful component life. The remaining useful life (\gls{rul}) of a system can be estimated based on history trajectory data, this approach which we refer here as data-driven can help improve maintenance schedules to avoid engineering failures and to save costs \citep{Lee2014}.

The existing \gls{phm} methods can be grouped into three different categories: model-based \citep{Yu2001} , data-driven \citep{Liu2009, Mosallam2013} and hybrid approaches \citep{Pecht2010, Liu2012}.

Model-based approaches attempt to incorporate physical models of the system into the estimation of the \gls{rul}. If the system degradation is modeled  precisely, model-based approaches usually exhibit better performance than data-driven approaches \citep{Qian2017}, nevertheless this comes at the expense of having extensive a priori knowledge of the underlying system and having a fine-grained model of such system (which usually involve expensive computations). On the other hand, data-driven approaches tend to use pattern recognition to detect changes in system states. Data-driven approaches are appropriate when the understanding of first principles of system operation is not comprehensive or when the system is sufficiently complex (i.e. jet engines, car engines, complex machinery) such that developing an accurate model is prohibitively expensive. Common disadvantages for the data-driven approaches are that they usually exhibit wider confidence intervals than model-based approaches and that a fair amount of data is required for training. Many data-driven algorithms have been proposed and good prognostics results have been achieved, among the most popular algorithms we can find artificial neural networks (\gls{ann}) \citep{Gebraeel2004}, support vector machine (\gls{svm}) \citep{Benkedjouh2013} and Markov hidden chains (\gls{mhc}) \citep{Dong2007}.

Over the past few years, data-driven approaches have gained more attention in the \gls{phm} community. A number of machine learning techniques, especially neural networks have been successfully applied to estimate the \gls{rul} of diverse mechanical systems. \glspl{ann} have demonstrated good performance when applied for modeling highly nonlinear, complex, multi-dimensional system without any prior expertise on the system's physical behavior \citep{Li2018}. While the confidence limits for the \gls{rul} predictions can not be naturally provided \citep{Sikorska2011}, the neural network approaches are promising on prognostic problems.

Neural networks for estimating the \gls{rul} of jet engines has been previously explored in \citep{Lim2016} where the authors propose a multi-layer perceptron (\gls{mlp}) coupled with a feature extraction (\gls{fe}) method and a time window for the generation of the features for the \gls{mlp}. In the publication the authors demonstrate that a moving window combined with a suitable feature extractior can improve the \gls{rul} prediction reported by other similar methods in the literature. In \citep{Li2018} the authors explore an even newer \gls{ann} architecture, the so-called convolutional neural networks (\glspl{cnn}), where they demonstrate that by using a \gls{cnn} without any pooling layers coupled with a time window the predicted \gls{rul} is further improved.

In this paper, we propose a novel framework for estimating the \gls{rul} of complex mechanical systems. The framework consists of a \gls{mlp} to estimate the \gls{rul} of the system at hand, coupled with an evolutionary algorithm for the fine tuning of data-related parameters, i.e. parameters that define the shape and quality of the features used by the \gls{mlp}. The publicly available NASA \gls{cmaps} dataset \citep{CMAPS2008} is used to assess the efficiency and reliability of the proposed framework. This approach allows for simple and small \glspl{mlp} to obtain better results than those reported in the current literature while using less computing power. 

The remainder of this paper is organized as follows: The \gls{cmaps} dataset is presented in Section \ref{sec:rul_dataset}, then the framework and all of its components are thoroughly reviewed in Section \ref{sec:method}. The method is evaluated using the \gls{cmaps} dataset in Section \ref{sec:rul_eval}, a comparison with the state-of-the-art is also provided. Finally, our conclusions are presented in Section \ref{sec:conclusions}.