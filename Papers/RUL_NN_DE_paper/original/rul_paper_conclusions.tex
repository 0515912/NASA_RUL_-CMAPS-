\section{Conclusions and future work}
\label{sec:conclusions}

This paper presents a novel framework for predicting the \gls{rul} of mechanical components. While the method was tested on the jet-engine specific dataset \gls{cmaps}, the method is general enough so that it can theoretically be applied to other kind of similar systems. The framework makes use of a strided moving time window to generate the training and test sets, a shallow \gls{mlp} to make the predictions of the \gls{rul} and an evolutionary algorithm (\gls{de}) which needs to be run just once in order to find the best data-related parameters that optimize the scoring functions used in this study.  The results presented in this paper demonstrate that the proposed framework is accurate and computationally efficient, which makes this framework suitable for applications that have limited computational resources such as embedded systems. Furthermore, a comparison with other state-of-the-art methods shown that the proposed method is the best overall performer. 

Two major features of the proposed framework are its generality and scalability. While for this paper very specific regressors and evolutionary algorithms were chosen, many other combinations are possible and may be more suitable for different applications. Furthermore, the framework here presented can, in principle, be used for model-construction, i.e. generating the best possible neural network architecture tailored to a specific application. Both issues are to be addressed in future work.