\section{Introduction}
\label{sec:rul_intro}

Traditionally, maintenance of mechanical systems has been carried out based on scheduling strategies, nevertheless strategies such as breakdown corrective maintenance and scheduled preventive maintenance are often costly and less capable of meeting the increasing demand of efficiency and reliability \cite{Gebraeel2005, Zaidan2013}. Condition Based Maintenance (\gls{cbm}) also known as intelligent Prognostics and Health Management (\gls{pmh}) allows for maintenance based on the current health of the system, thus cutting costs and increasing the reliability of the system \cite{Zhao2017}. To avoid confusion, here we define prognostics as the estimation of remaining useful component life. The Remaining Useful Life (\gls{rul}) of a system can be estimated based on history trajectory data, this approach which we refer here as data-driven can help improve maintenance schedules to avoid engineering failures and save costs \cite{Lee2014}.

The existing \gls{pmh} methods can be grouped into three different categories: model-based \cite{Yu2001} , data-driven \cite{Liu2009, Mosallam2013} and hybrid approaches \cite{Pecht2010, Liu2012}.

Model-based approaches attempt to incorporate physical models of the system into the estimation of the \gls{rul}. If the system degradation is modeled  precisely, model-based approaches usually exhibit better performance than data-driven approaches \cite{Qian2017}, nevertheless this comes at the expense of having extensive a prior knowledge of the underlying system and having a fine-grained model of such system (which usually involve expensive computations). On the other hand data-driven approaches tend to use pattern recognition to detect changes in system states. Data-driven approaches are appropriate when the understanding of first principles of system operation is not comprehensive or when the system is sufficiently complex (i.e. jet engines, car engines, complex machinery) such that developing an accurate model is prohibitively expensive. Common disadvantages for the data-driven approaches are that they usually exhibit wider confidence intervals than model-based approaches and that a fair amount of data is required for training. Many data-driven algorithms have been proposed and good prognostics results have been achieved, among the most popular algorithms we can find Artificial Neural Networks (\gls{ann}) \cite{Gebraeel2004}, Support Vector Machine (\gls{svm}) \cite{Benkedjouh2013}, Markov Hidden Chains (\gls{mhc}) \cite{Dong2007}.

Over the past few years, data-driven approaches have gained more attention in the \gls{pmh} community. A number of machine learning techniques, especially neural networks have been successfully applied to the estimate \gls{rul} of diverse mechanical systems. \glspl{ann} have demonstrated good performance when applied for modeling highly nonlinear, complex, multi-dimensional system without any prior expertise on the system's physical behavior \cite{Li2018}. While the confidence limits for the \gls{rul} predictions can not be naturally provided \cite{Sikorska2011}, the neural network approaches are promising on prognostic problems.

In this paper we propose a Multi-layer Perceptron (\gls{mlp}) architecture coupled with a strided time-window approach for estimating the \gls{rul} of aero-engines. The publicly available NASA \gls{cmaps} dataset \cite{CMAPS2008}. Raw sensor measurements with normalization are directly used as inputs to the \gls{mlp} which then outputs the \gls{rul} of the jet engine in terms of cycles. 

The use of Neural Networks for estimating the \gls{rul} of jet engines has been previously explored in \cite{Lim2016} where the authors propose a Multi-layer Perceptron \gls{mlp} coupled with a Feature Extraction (\gls{fe}) method and a time window for the generation of the features for the \gls{mlp}. In the publication the authors demonstrate that a moving window combined with a moving time-window and suitable feature extraction they can improve the \gls{rul} prediction reported by other similar methods in the literature. In \cite{Li2018} the authors explore an even newer \gls{ann} architecture, the so-called Convolutional Neural Networks \glspl{cnn}, where they demonstrate that by using a \gls{cnn} without any pooling layers coupled with a time-window as described in \cite{Lim2016} cite the predicted \gls{rul} is further improved.

The present work takes inspiration from \cite{Lim2016} and \cite{Li2018} in the sense that an \gls{ann} architecture coupled with a time-window is used to produce the \gls{rul} predictions, nevertheless this research also considers the use of a \textit{strided} time-window which allows for more accurate predictions of the \gls{rul} of the jet-engine than the reported in \cite{Lim2016} and \cite{Li2018}. Furthermore, this paper presents an optimization framework, based on the Root Mean Squared Error \gls{rmse} of the predictions, for the fine tuning of data related hyperparameters such as window-size, stride-size, etc. Such approach allows a simple \gls{mlp} to obtain even better results than those reported in the current literature using less computing power.

The remainder of this paper is organized as follows:...