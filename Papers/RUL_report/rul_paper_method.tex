\section{Estimating Remaining Useful Life using Multi-Layer Perceptron as Regressor}
\label{sec:method}

In this section the proposed \gls{ann}-based method for prognostics is presented. As stated earlier this work is highly influenced by the ideas presented in \cite{Lim2016} and \cite{Li2017}. In short our method uses a Multi-Layer Perceptron (\gls{mlp}) as the main regressor for estimating the \gls{rul} of the engines at each subset of the \gls{cmaps} dataset. Each subset is preprocessed before feeding it into the Neural Network. In this work we adopt a similar preprocessing of the data as the one described in \cite{Lim2016}. A time window is used to better capture the degrading effect of the fault at the sensor readings of the engine, in a similar manner we adopt a constant \gls{rul} at the early cycles of the engine and then model its degradation using a linear model \cite{Li2017}. In \cite{Li2017} and \cite{Lim2016} the moving time window only updates one point at the time,  this paper introduces the use of a \textit{strided} time window capable of updating $n_s$ points at the time. As can be noted the shape and ``quality'' of the data is dependent upon 3 hyperparameters, namely the window size $n_w$, the constant \gls{rul} $C_{RUL}$, used at the early ages of the engine and the window stride $n_s$. Instead of tuning these hyperparameters by hand where propose to optimize them using a meta-heuristic method, for this study the method called Differential Evolution (\gls{de}) was used.