\section{NASA C-MAPSS Dataset}
\label{sec:rul_dataset}

The NASA \gls{cmaps} dataset \cite{CMAPS2008} is used to evaluate the proposed method. The \gls{cmaps} dataset contains simulated data produced using a model based simulation program (Commercial Modular Aero-Propulsion System Simulation) developed by NASA. The dataset is further divided into 4 subsets composed of multi-variate temporal data obtained from 21 sensors.

For each of the 4 subsets a training and a test set is provided. The training sets include run-to-failure sensor records of multiple aero-engines collected under different operational conditions and fault modes as described in Table \ref{table:cmapss}

The data is arranged in an $n\times26$ matrix where $n$ corresponds to the number of data points in each subset. The first two variables represent the engine and cycle numbers respectively. The following three variables are operational settings which correspond to the operating conditions in Table \ref{table:cmapss} and have a substantial effect on engine performance. The remaining variables represent the 21 sensor readings that model the engine degradation throughout time.

\begin{table}[!htb]
\centering
\begin{tabular}{l | l l l l}
	\hline
	 & \multicolumn{4}{c}{C-MAPSS}\\  
	 Dataset & FD001 & FD002 & FD003 & FD004\\
  	\hline
  	Train Trajectories & 100 & 260 & 100 & 248\\
  	Test Trajectories & 100 & 259 & 100 & 248\\
  	Operating Conditions & 1 & 6 & 1 & 6\\
  	Fault Modes & 1 & 1 & 2 & 2\\
  	\hline
\end{tabular}
\caption{C-MAPSS Dataset details}
\label{table:cmapss}
\end{table}

Each trajectory within the train and test trajectories is assumed to represent the life-cycle of an engine. Each engine is simulated with different initial health conditions (no faults). For each trajectory of an engine the last data entry corresponds to the moment the engine is declared faulty. On the other hand the trajectories within the test sets terminate at some point prior to failure  and the aim is to predict the Remaining Useful Life (\gls{rul}) of each engine in the test set. The actual \gls{rul} value of each test trajectories were also included in the dataset for verification purposes. Further discussion of the dataset and details on how the data is generated are given in \cite{Saxena2008}.

\subsection{Performance evaluation}
\label{sec:rul_metrics}

The evaluate the performance of the proposed approach on the \gls{cmaps} dataset we make use of two scoring indicators, namely the Root Mean Squared Error (\gls{rmse}) and a score function proposed in \cite{Saxena2008} which we refer in this work as \gls{rul} Health Score (\gls{rhs}). 

\begin{equation}
RMSE = \sqrt{ \frac{1}{N} \sum_{i=1}^{N}{d_i^2}}
\end{equation}

\begin{align}
s &= \sum_{i=1}^{N}{s_i}\\
s_i &= \begin{cases} 
      e^{-\frac{d_i}{13}} - 1 & d_i < 0, \\
      e^{-\frac{d_i}{10}} - 1 & d_i \geq 0
\end{cases}
\end{align}

where $s$ denotes the score and $N$ is the total number of testing data samples. $d_i = RUL_i^p - RUL_i$, that is the error between the estimated \gls{rul} value and the actual \gls{rul} value for the \textbf{i-th} testing sample. The (\gls{rhs}) function penalizes late predictions more than early predictions since usually late predictions lead to more severe consequences in fields such as aerospace.

