%%%%%%%%%%%%%%%%%%%%%%%%%%%%%%%%%%%%%%%%%
% Journal Article
% LaTeX Template
% Version 1.3 (9/9/13)
%
% This template has been downloaded from:
% http://www.LaTeXTemplates.com
%
% Original author:
% Frits Wenneker (http://www.howtotex.com)
%
% License:
% CC BY-NC-SA 3.0 (http://creativecommons.org/licenses/by-nc-sa/3.0/)
%
%%%%%%%%%%%%%%%%%%%%%%%%%%%%%%%%%%%%%%%%%

%----------------------------------------------------------------------------------------
%	PACKAGES AND OTHER DOCUMENT CONFIGURATIONS
%----------------------------------------------------------------------------------------

\documentclass{article}

\usepackage{mathtools} %tools for mathematical writing
\usepackage{caption}
\usepackage{subfig}
\usepackage{float}
\usepackage{color}
\usepackage{adjustbox}
\usepackage{comment}

\usepackage[sc]{mathpazo} % Use the Palatino font
\usepackage[T1]{fontenc} % Use 8-bit encoding that has 256 glyphs
\linespread{1.05} % Line spacing - Palatino needs more space between lines
\usepackage{microtype} % Slightly tweak font spacing for aesthetics

\usepackage[hmarginratio=1:1,top=32mm,columnsep=20pt]{geometry} % Document margins
%\usepackage{multicol} % Used for the two-column layout of the document
%\usepackage[hang, small,labelfont=bf,up,textfont=it,up]{caption} % Custom captions under/above floats in tables or figures
\usepackage{booktabs} % Horizontal rules in tables
%\usepackage{float} % Required for tables and figures in the multi-column environment - they need to be placed in specific locations with the [H] (e.g. \begin{table}[H])
\usepackage{hyperref} % For hyperlinks in the PDF

\usepackage{lettrine} % The lettrine is the first enlarged letter at the beginning of the text
\usepackage{paralist} % Used for the compactitem environment which makes bullet points with less space between them

\usepackage{abstract} % Allows abstract customization
\renewcommand{\abstractnamefont}{\normalfont\bfseries} % Set the "Abstract" text to bold
\renewcommand{\abstracttextfont}{\normalfont\small\itshape} % Set the abstract itself to small italic text

\usepackage{titlesec} % Allows customization of titles
%\renewcommand\thesection{\Roman{section}} % Roman numerals for the sections
%\renewcommand\thesubsection{\Roman{subsection}} % Roman numerals for subsections
\titleformat{\section}[block]{\large\scshape\centering}{\thesection.}{1em}{} % Change the look of the section titles
\titleformat{\subsection}[block]{\large}{\thesubsection.}{1em}{} % Change the look of the section titles

\usepackage{fancyhdr} % Headers and footers
\pagestyle{fancy} % All pages have headers and footers
\fancyhead{} % Blank out the default header
\fancyfoot{} % Blank out the default footer
\fancyhead[C]{Running title $\bullet$ April 2018 $\bullet$ Vol. XXI, No. 1} % Custom header text
\fancyfoot[RO,LE]{\thepage} % Custom footer text

\usepackage{algorithm}% http://ctan.org/pkg/algorithms
\usepackage{algpseudocode}% http://ctan.org/pkg/algorithmicx

\usepackage[acronym]{glossaries} %Used for the acronyms

\DeclarePairedDelimiter\ceil{\lceil}{\rceil}
\DeclarePairedDelimiter\floor{\lfloor}{\rfloor}

%My definitions
%\renewcommand{\tablename}{Table}

\newtheorem{theorem}{Theorem}
\newtheorem{proposition}{Proposition}[section]
\newtheorem{corollary}{Corollary}[theorem]
\newtheorem{remark}{Remark}
\newenvironment{proof}{\begin{trivlist} \item[]\textbf{Proof}
\hspace{0cm} }{\hfill $\Box$ \end{trivlist}}
\newtheorem{mydef}{Definition}[section]
%\newcommand{\eqref}[1]{\mbox{(\ref{#1})}}
\newcommand{\Tr}{\mbox{Tr}}

%Set space between paragraphs
\setlength{\parskip}{2ex}

\usepackage{authblk}

\providecommand{\keywords}[1]{\textbf{\textit{Index terms---}} #1}
%----------------------------------------------------------------------------------------
%	TITLE SECTION
%----------------------------------------------------------------------------------------

\title{Technical Report on the solution of the CMAPSS-RUL dataset using Neural Networks}
\author[1]{David Laredo\thanks{dlaredorazo@ucmerced.edu}}
%\author[2]{Honggang Wang\thanks{honggang.w@rutgers.edu}}
\author[1]{Jian-qiao Sun\thanks{jqsun3@ucmerced.edu}}
\affil[1]{School of Mechanical Engineering, University of California, Merced}
%\affil[2]{Department of Systems \& Industrial Engineering, Rutgers University, New Jersey, USA}

\date{}
%\renewcommand\Authands{ and }

% Generate the glossary
\makeglossaries
\renewcommand*{\acronymname}{Acronyms}

%----------------------------------------------------------------------------------------

\begin{document}

\maketitle % Insert title

\thispagestyle{fancy} % All pages have headers and footers

%Term definitions
\newacronym{rul}{RUL}{Remaining Useful Life}
\newacronym{mlp}{MLP}{Multi-layer Perceptron}
\newacronym{cmaps}{C-MAPSS}{Commercial Modular Aero Propulsion System Simulator}
\newacronym{ann}{ANN}{Arificial Neural Networks}
\newacronym{cbm}{CBM}{Condition Based Maintenance}
\newacronym{pmh}{PMH}{Prognostics and Health Management}
\newacronym{ml}{ML}{Machine Learning}



%Remove after the paper is complete.
\newacronym{moo}{MOO}{Multi-objective Optimization}
\newacronym{mop}{MOP}{Multi-objective Optimization Problem}
\newacronym{mmop}{MMOP}{Mixed-Integer Multi-objective Optimization Problem}
\newacronym{eds}{EDS}{Enhanced Directed Search}
\newacronym{dzz}{DZZ}{Direct Zig Zag}
\newacronym{nsga2}{NSGA-II}{Non-Sorted Genetic Algorithm II}
\newacronym{sop}{SOP}{Single-objective Optimization Problem}
\newacronym{pc}{PC}{Predictor-Corrector}
\newacronym{moea}{MOEA}{Multi-objective Optimization Evolutionary Algorithm}
\newacronym{ea}{EA}{Evolutionary Algorithm}
\newacronym{ds}{DS}{Directed Search}
\newacronym{kkt}{KKT}{Karush-Kuhn-Tucker}
\newacronym{bop}{BOP}{Bi-objective Optimization Problem}
\newacronym{gd}{GD}{Generational Distance}
\newacronym{igd}{IGD}{Inverted Generational Distance}
\newacronym{gsa}{GSA}{Gradient Subspace Approximation}
\newacronym{ift}{IFT}{Implicit Function Theorem}
\newacronym{fps}{FPS}{First Pareto Solution}
\newacronym{moead}{MOEA-D}{Multi-objective Optimization Evolutionary Algorithm based on Decomposition}
\newacronym{pf}{PF}{Pareto Front}
\newacronym{nbi}{NBI}{Normal Boundary Intersection}
\newacronym{pso}{PSO}{Particle Swarm Optimization}
 %Insert acronym list

%Used to not expand the first acronym
\glsunsetall

%----------------------------------------------------------------------------------------
%	ABSTRACT
%----------------------------------------------------------------------------------------

\begin{abstract}

\noindent 

In this report we present an a data-driven approach for estimating the Remaining Useful Life (\gls{rul}) of aero-engines. A ``strided'' time window approach is employed to generate training and test sets to be used with a conventional Multi-layer Percepton (\gls{mlp}) which will serve as the main regressor for this application, no model for the engine is required . The proposed approach is evaluated on the publicly available \gls{cmaps} dataset. The accuracy of the proposed method is compared against other state-of-the art methods available in the literature. 
\end{abstract}

\keywords{Artificial Neural Networks (\gls{ann}), Moving Time Window, \gls{rul} Estimation, \gls{cmaps}, Prognostics}

%----------------------------------------------------------------------------------------
%	ARTICLE CONTENTS
%----------------------------------------------------------------------------------------

\section{Introduction}
\label{sec:rul_intro}

Traditionally, maintenance of mechanical systems has been carried out based on scheduling strategies, nevertheless such strategies are often costly and less capable of meeting the increasing demand of efficiency and reliability \cite{Gebraeel2005, Zaidan2013}. Condition Based Maintenance (\gls{cbm}) also known as intelligent Prognostics and Health Management (\gls{pmh}) allows for maintenance based on the current health of the system, thus cutting costs and increasing the reliability of the system \cite{Zhao2017}. To avoid confusion, here we define prognostics as the estimation of remaining useful component life. The Remaining Useful Life (\gls{rul}) of a system can be estimated based on history trajectory data, this approach which we refer here as data-driven can help improve maintenance schedules to avoid engineering failures and save costs \cite{Lee2014}.

The existing \gls{pmh} methods can be grouped into three different categories: model-based \cite{Yu2001} , data-driven \cite{Liu2009, Mosallam2013} and hybrid approaches \cite{Pecht2010, Liu2012}.

Model-based approaches attempt to incorporate physical models of the system into the estimation of the \gls{rul}. If the system degradation is modeled  precisely, model-based approaches usually exhibit better performance than data-driven approaches \cite{Qian2017}, nevertheless this comes at the expense of having extensive a prior knowledge of the underlying system and having a fine-grained model of such system (which usually involve expensive computations). On the other hand data-driven approaches tend to use pattern recognition to detect changes in system states. Data-driven approaches are appropriate when the understanding of first principles of system operation is not comprehensive or when the system is sufficiently complex (i.e. jet engines, car engines, complex machinery) such that developing an accurate model is prohibitively expensive. Common disadvantages for the data-driven approaches are that they usually exhibit wider confidence intervals than model-based approaches and that a fair amount of data is required for training. Many data-driven algorithms have been proposed and good prognostics results have been achieved, among the most popular algorithms we can find Artificial Neural Networks (\gls{ann}) \cite{Gebraeel2004}, Support Vector Machine (\gls{svm}) \cite{Benkedjouh2013}, Markov Hidden Chains (\gls{mhc}) \cite{Dong2007}.

Over the past few years, data-driven approaches have gained more attention in the \gls{pmh} community. A number of machine learning techniques, especially neural networks have been successfully applied to the estimate \gls{rul} of diverse mechanical systems. \glspl{ann} have demonstrated good performance when applied for modeling highly nonlinear, complex, multi-dimensional system without any prior expertise on the system's physical behavior \cite{Li2018}. While the confidence limits for the \gls{rul} predictions can not be naturally provided \cite{Sikorska2011}, the neural network approaches are promising on prognostic problems.

Neural Networks for estimating the \gls{rul} of jet engines has been previously explored in \cite{Lim2016} where the authors propose a Multi-layer Perceptron \gls{mlp} coupled with a Feature Extraction (\gls{fe}) method and a time window for the generation of the features for the \gls{mlp}. In the publication the authors demonstrate that a moving window combined with a suitable feature extractior can improve the \gls{rul} prediction reported by other similar methods in the literature. In \cite{Li2018} the authors explore an even newer \gls{ann} architecture, the so-called Convolutional Neural Networks \glspl{cnn}, where they demonstrate that by using a \gls{cnn} without any pooling layers coupled with a time-window the predicted \gls{rul} is further improved.

In this paper we propose a novel framework for estimating the \gls{rul} of complex mechanical systems. The framework consists of a Multi-layer Perceptron (\gls{mlp}) to estimate the \gls{rul} of the system at hand, coupled with an evolutionary algorithm for the fine tuning of data-related parameters, i.e. parameters that define the shape and quality of the features used by the \gls{mlp}. The publicly available NASA \gls{cmaps} dataset \cite{CMAPS2008} is used to assess the efficiency and reliability of the proposed framework. This approach allows for even simple and small \glspl{mlp}, thus being suitable for computationally restricted environments, to obtain better results than those reported in the current literature while using less computing power.

The remainder of this paper is organized as follows: The \gls{cmaps} dataset is presented in Section \ref{sec:rul_dataset}, then the framework and all of its components are thoroughly reviewed in Section \ref{sec:method}. The method is evaluated using the \gls{cmaps} dataset in Section \ref{sec:rul_eval}, a comparison with the state-of-the-art is also provided. Finally, our conclusions are presented in Section \ref{sec:conclusions}.
\section{Problem Statement and Background}
\label{sec:background}

A multi-objective optimization problem (\gls{mmop}) can be formally stated as:

\begin{equation}
	\begin{aligned}
	\underset{x \in \mathbb{R}^n}{\text{min}}
	& \quad F(x)\\
	\text{s. t.} \\
	& \quad h_i(x) = 0, \; i = 1 \ldots m\\
	& \quad h_j(x) \leq 0, \; j = 1 \ldots p,\\
	\end{aligned}
	\label{eq:mop}
\end{equation}

\noindent where $F:\mathbb{R}^n \rightarrow \mathbb{R}^k$, $F(x) = (f_1(x), \ldots,f_k(x))^T$ represents a vector of $k \geq 2$ \emph{objective functions}. The \emph{feasible decision vectors}, that form the set $\mathbb{X}$, are those $x \in \mathbb{R}^{n}$ that comply with the equality $h_i(x)$ and inequality $h_j(x)$ constraints.\\

The optimality of a \gls{mop} is defined by the concept of dominance \cite{pareto_set}.

\begin{mydef}[Pareto Dominance]
A point $y \in \mathbb{X}$ is dominated by a point  $x \in \mathbb{X} \; (x \prec y)$ with respect to Eq. \eqref{eq:mop} if $x$ is partially less than $y$, i.e., if $f_{i}(x) \leq f_{i}(y)$, for all $i \in 1, \ldots, k$, and $f_{j}(x) < f_{j}(y)$ for some $j \in 1, \ldots, k$. Otherwise it is non-dominated by $x$. 
\end{mydef}

\begin{mydef}[Pareto Optimality]
A decision vector $x^{*} \in \mathbb{X}$ is Pareto optimal with respect to Eq. \eqref{eq:mop} if there does not exist another decision vector $x \in \mathbb{X}$ such that  $x \prec x^{*}$.
\end{mydef}

\begin{mydef}[Pareto Weak Optimality]
A decision vector $x^{*} \in \mathbb{X}$ is \emph{weakly} Pareto optimal with respect to Eq. \eqref{eq:mop} if there does not exist another decision vector $x \in \mathbb{X}$, such that  $f_i(x) < f_i(x^{*}) \quad \forall i = 1, ... , k$.
\end{mydef}

In general, the solution of a \gls{mop} consists not only of a single solution but of a set of solutions which have to be considered as optimal. The solution set is called the \emph{Pareto set} and its corresponding image is called the \emph{Pareto front} which typically forms a ($k-1$)-dimensional manifold \cite{hillermeier01}, where $k$ is the number of objectives involved in the problem. The concepts of Pareto set and Pareto front are formalized in the following definition:

\begin{mydef}[Pareto set and Pareto front]
The set of optimal points $\mathcal{P}$ for Eq. \eqref{eq:mop}, 
\[
\mathcal{P} = \{ x \in \mathbb{X} \ |\not \exists y  \in \mathbb{X} : y \prec x \}
\]

\noindent is called the Pareto set. The image $F(\mathcal{P})$ is called the Pareto front.
\end{mydef}

The Jacobian of $F$ at a point $x$ is given by 

\begin{equation}
J(x) = \left(\begin{array}[]{c}\nabla f_{1}\\
\vdots\\
\nabla f_{m}\end{array}\right) \in \mathbb{R}^k.
\label{eq:jacobian}
\end{equation}

where $\nabla f_i(x)$ denotes the gradient of objective $f_i$. In case all the objectives of the \gls{mop} are differentiable the following famous theorem of Kuhn and Tucker \cite{kkt_conditions} states a necessary condition for Pareto optimality for unconstrained \glspl{mop}.

\begin{theorem}[KKT Conditions]
\label{theo:kkt_cond}
Let $x^*$ be a Pareto point of problem \ref{eq:mop}, then there exists a vector $\alpha \in R^k$ with $\alpha_i \geq 0$, $i = 1,\ldots,k$, and $\sum_{i = 1}^k$ such that

\begin{equation}
\sum_{i = 1}^k \alpha_i \nabla f_i(x^*) = J(x)^T \alpha = 0.
\label{eq:alpha_vector}
\end{equation}
\end{theorem}

Points satisfying Eq. \ref{eq:alpha_vector} are called \emph{Karush-Kuhn-Tucker (\gls{kkt})} points. One important thing to outline is that given a \gls{kkt} point $x^*$ its associated weight vector $\alpha$ is normal to the linearization (tangent) of the Pareto front at $F(x^*)$. It can also be noted that $rank (J(x^*)^T) < k$ for any $x$ that is a \gls{kkt} point \cite{hillermeier01}.

\subsection{Mixed-Integer Optimization}
\label{sec:mixed_integer_mops}

A mixed-integer multi-objective optimization problem (\gls{mmop}) can be formally stated as:

\begin{equation}
	\begin{aligned}
	\underset{x  \in \mathbb{Z}^{d_{1}} \times \mathbb{R}^{d_{2}}}{\text{min}}
	& \quad F(x)\\
	\text{s. t.}\\
	& \quad h_i(x) = 0, \; i = 1 \ldots m\\
	& \quad h_j(x) \leq 0, \; j = 1 \ldots p,\\
	\end{aligned}
	\label{eq:dmop2}
\end{equation}

\noindent where $F:\mathbb{Z}^{d_1} \times \mathbb{R}^{d_2} \to \mathbb{R}^k$, which means that the parameter vector can be formed either by real variables, discrete (or integer) variables or a mixture of both depending on the values of $d_1$ and $d_2$ respectively. For instance, Figure \ref{fig:spaces_int} displays a two dimensional integer space where $x_1, x_2 \in \mathbb{Z}$, Figure \ref{fig:spaces_mix} represents a mixed-integer space where $x_1 \in \mathbb{Z}$ and $x_2 \in \mathbb{R}$,  while in Figure \ref{fig:spaces_real} $x_1, x_2 \in \mathbb{R}$, that is, they belong to a real space.

The goal of Eq. \eqref{eq:dmop2}, as in the continuous case, is to seek non-dominated (Pareto optimal) solutions of the objective function $F$ on the feasible set $\mathbb{X}$.

\begin{comment}

\begin{figure}[H]
    \subfloat[Integer space \label{fig:spaces_int}]{%
    	\centering \def\svgwidth{120pt} 
		\input{img/integer_space.pdf_tex} 
    }
    \hfill
    \subfloat[Mixed-Integer space \label{fig:spaces_mix}]{%
    	\centering \def\svgwidth{120pt} 
		\input{img/mixed_integer_space.pdf_tex} 
    }
     \hfill
    \subfloat[Real space \label{fig:spaces_real}]{%
      \centering \def\svgwidth{120pt} 
		\input{img/real_space.pdf_tex} 
    }
    \caption{Examples of decision spaces}
    \label{fig:spaces_mixed_integer}
\end{figure}

\end{comment}

In this work the theory developed for continuous \glspl{mop} will be used for the treatment of \glspl{mmop} since it can be shown, that under some assumptions, most of theory presented within this chapter holds for \glspl{mmop}.

\subsection{Direct Zig Zag Method}
\label{sec:dzz_method}

The Direct Zig-Zag (\gls{dzz}) method \cite{zigzag_discrete} is a continuation method for solving discrete or mixed-integer bi-objective problems. The \gls{dzz} method searches Pareto optimal solutions along a zig-zag path close to the Pareto front. The local zig-zag path is identified based on a pattern search idea (e.g. Hooke-Jeeves method \cite{hooke_jeeves}) in which the search procedure only compares function values without computing the gradients of the objective functions. Thus, the \gls{dzz} method can, in general, be applied for black-box discrete or mixed-integer \glspl{mop} where the objective functions can be evaluated through numerical or simulation processes. The \gls{dzz} method guarantees local Pareto optimality of the solutions due to the neighborhood search inside the pattern search procedure. The method consists of two parts.

In the first part, a First Pareto Solution (\gls{fps}) is computed, from a starting point $x_0$, by means of a modified version of the well known pattern search method \cite{hooke_jeeves}. This \gls{fps} is the Pareto point which minimizes $f_1$ while attaining the smallest value of $f_2$.

The second part is performed in an iterative way. For each Pareto solution $x_0^*$ the method looks for a neighboring solution $x_1$ that increases the value of $f_1$, i.e., $f_1(x_1) > f_1(x_0^*)$, this is called a \textit{zig step}. Since it is assumed that the continuation will start  at the Pareto point that minimizes $f_1$ (\gls{fps}) it is logical to think that the only direction to keep moving is the one that increases the values of $f_1$.  From $x_1$ a pattern search like strategy is applied in order to find a non-dominated solution $x_1^*$, such that $x_1^* \neq x_0^*$, this is a \textit{zag step}. Figure \ref{Fig:dzz_example} displays a simple example of the application of the \gls{dzz} method. The first phase, namely the \gls{fps} phase goes from $f_(x_0)$ to $f(x_1)$. From $f(x_1)$ onwards, zig and zag steps are iteratively applied.

\begin{comment}

\begin{figure}[H]
\centering \includegraphics[width = 60mm, height = 60mm]{img/dzz_method.png}
\caption{DZZ method example}
\label{Fig:dzz_example}
\end{figure}

\end{comment}

It is remarkable that the method does not use gradient information making it very efficient in terms of function evaluations. Nevertheless, it is only able to solve bi-objective problems so far, making this its greatest disadvantage.

\subsection{The Directed Search Method}
\label{sec:directed_search}

This method defines a way to steer the search for continuous \glspl{mop} by using a direction in objective space and mapping it into parameter space  \cite{directed_search}. Some of the concepts used by this method are quite important for the development of the \gls{eds}, thus a brief explantion of the key concepts of the \gls{ds} method are given next. The main idea is as follows.\\

Assume a point $x_0 \in \mathbb{R}^n$, in parameter space, with $rank(J(x_0)) = k$ and a vector $d \in \mathbb{R}^k$ representing a desired search direction in image space are given. Then, a search direction $\nu \in \mathbb{R}^n$ in decision space is sought such that for $y_0 := x_0 + t\nu$, where $t \in \mathbb{R}_+$ is the step size (i.e., $y_0$ represents a movement from $x_0$ in direction v), it holds:

\begin{equation}
	 \lim_{t \to 0} \frac{f_i(y_0) - f_i(x_0)}{t} = \langle \nabla f_i(x_0), \nu \rangle = d_i, \quad i = 1,...,k.\\
	 \label{eq:direction1}
\end{equation}

Using the Jacobian of $F$, Eq. \eqref{eq:direction1} can be stated in matrix vector notation as

\begin{equation}
	J(x_0)\nu = d.
	\label{eq:direction2}
\end{equation}

Hence, such a search direction $\nu$ can be computed by solving a system of linear equations. Since typically the number of decision variables is (much) higher than the number of objectives for a given \gls{mop}, i.e., $n \gg k$, system \eqref{eq:direction2} is (probably highly) underdetermined, which implies that its solution is not unique. One possible choice is to take

\begin{equation}
	\nu_+ = J(x_0)^+d,
	\label{eq:direction3}
\end{equation}

where $ J(x_0)^+ \in \mathbb{R}^{n \times k}$ denotes the pseudo inverse\footnote{If the rank of $J := J(x_0)$ is $k$ (i.e., maximal) the pseudo inverse is given by $J^+ = J^T(JJ^T)^{-1}$.} of $J(x_0)$. A new iterate $x_1$ can be computed as the following discussion shows: given a candidate solution $x_0$, a new solution is obtained via $x_1 = x_0 + t\nu$, where $t > 0$ is a step size and $\nu \in \mathbb{R}^n$ is a vector that satisfies \eqref{eq:direction2}. Among the solutions of system \eqref{eq:direction2}, $\nu_+$ is the one with the smallest Euclidean norm. Hence, given $t$, one expects for a step in direction $\nu_+$ (decision space) the largest progress in $d$-direction (objective space).

Given a direction $d \in \mathbb{R}^k \texttt{\char`\\} \{0\}$ with $d_i \leq 0, i = 1, \ldots, k$, a point $x_0 \in \mathbb{R}^n$ with $\text{rank}(J(x_0)) = k$ and assuming that the image of $F$ is bounded from below, a greedy search in direction $d$ using Eq. \eqref{eq:direction3} leads to the (numerical) solution of the following initial value problem:

\begin{eqnarray}
		\label{eq:ds_initial_value_problem}
		x(0) & = & x_0 \in \mathbb{R}^n\\
		\dot{x}(t) & = & J(x(t))^+d,  \quad t > 0 \nonumber.
\end{eqnarray}

\vfill
\newpage

\begin{mydef}[Critical Point]
\label{def:critical_point}

Let $\gamma:[0, t_f] \to \mathbb{R}^n$ be a solution of Eq. \eqref{eq:ds_initial_value_problem} and let $t_c$ be the smallest value of $t > 0$ such that

\begin{equation}
\not \exists v \in \mathbb{R}^n : \quad J(x(t))v = d.
\end{equation}.

Then $t_c$ and $\gamma(t_c)$ are a critical value an critical point of \eqref{eq:ds_initial_value_problem} respectively. 

\end{mydef}

By Definition \ref{def:critical_point}, it is possible to divide the solution $\gamma:[0, t_f]$ of Eq. \eqref{eq:ds_initial_value_problem} into two phases, (see Figure \ref{Fig:greedy_ds}):
	
	\begin{itemize}
		\item $\gamma([0, t_c])$: the function $F(\gamma(t))$ gets the desired decay in $d$-direction.
		\item $\gamma((t_c, t_f])$: the function $F(\gamma(t))$ moves along the critical points of $F$. For the end point $\gamma(t_f)$, it holds $J(\gamma(t_f))^+d = 0$.
	\end{itemize}
	
\begin{comment}	
	
\begin{figure}[H] 
	\centering \def\svgwidth{150pt} 
	\input{img/descent_ds.pdf_tex} 
	\caption{Greedy movement of DS descent} 
	\label{Fig:greedy_ds}
\end{figure}

\end{comment}	
	
The study made in \cite{directed_search} shows that \emph{critical points} are not necessarily \gls{kkt} points but are local solutions of the well known \gls{nbi} \cite{nbi} problem, therefore the \gls{ds} descent method is only restricted to the detection of such critical points. To trace the solution curve of Eq. \eqref{eq:ds_initial_value_problem} numerically, specialized \gls{pc} methods \cite{allgower01} can be used.\\

Using the ideas mentioned above, the authors of \cite{directed_search} developed a method to move towards and along the Pareto front of continuous \glspl{mop}, for a broader exposition of the \gls{ds} please refer to \cite{directed_search}.
\section{Estimating Remaining Useful Life using Multi-Layer Perceptron as Regressor}
\label{sec:method}

In this section the proposed \gls{ann}-based method for prognostics is presented. Our method uses a Multi-Layer Perceptron (\gls{mlp}) as the main regressor for estimating the \gls{rul} of the engines at each subset of the \gls{cmaps} dataset. For the training sets, the feature vectors are generated by using a strided time window while the labels vector is generated using a constant \gls{rul} for the early cycles of the simulation and then linearly decreasing the number of remaining cycles, this is the so called piecewise linear degradation model \cite{Ramasso2014}. For the test set, a time window is taken from the last sensor readings of the engine and used to predict the \gls{rul} of the engine.

The window size $n_w$, window stride $n_s$ and early \gls{rul} $R_e$ data related parameters have a considerable impact in the quality of the predictions made by the regressor. Hand picking the best parameters for our application is time consuming, furthermore, a grid search approach as the ones used for hyperparameter tuning in Neural Networks is computationally expensive given the search space inherent to the aforementioned parameters. In this paper we propose the use of an evolutionary algorithm, i.e. Differential Evolution (\gls{de}) \cite{Storn1997}, to fine tune the parameters. The optimization framework here proposed allows for the use of a simple Neural Network architecture while attaining better results in terms of the quality of the predictions made than the ones in the current literature.

\subsection{The Neural Network Architecture}

For this study we propose to use a rather simple \gls{mlp} architecture. All the implementations were used in python using the Keras/Tensorflow environment. The structure of the Network remained consisted for all the four datasets.

The choice of the network architecture was made using an iterative process; comparing 4 different architectures, running each $10$ times for $100$ epochs and using a mini-batch size of $512$. Two objectives were pursued; that the size of the architecture was small enough, e.g. in terms of layers and neurons within each layer and that the performance indicators were better than the ones presented in the literature so far. The process for choosing the network architecture can be described as follows: First, fix the window size $n_w$, the window stride $n_s$ and the early RUL $R_e$. An initial setup of two layers with $250$ and $50$ neurons each was chosen, the number of neurons at each layer was then reduced by roughly a factor of 2 and tested using a cross-validation set from subset 1 of \gls{cmaps}. When the performance of the network started to degrade the process was stopped. Table  \ref{table:tested_architectures_100} summarizes the results for each tested architecture, Table \ref{table:proposed_nn} presents the architecture chosen for the remainder of this work yielded the best results and hence was the chosen for the rest of the experiments. Details for the other 3 tested architectures are presented in Section ...

\begin{table}[!htb]
\centering
\begin{tabular}{l r r | r r | r r | r r}
	\hline	
	& \multicolumn{2}{c}{Min.} & \multicolumn{2}{c}{Max.}  & \multicolumn{2}{c}{Avg.}  & \multicolumn{2}{c}{STD} \\
	Tested Architecture & RMSE & RHS & RMSE & RHS & RMSE & RHS & RMSE & RHS\\
  	\hline
  	Architecture 1 & 10.85 & 151.65 & 12.23 & 277.67 & 11.66 & 226.94 & 0.45 & 41.58\\
  	Architecture 2 & 11.12 & 186.22 & 13.98 & 365.92 & 12.68 & 280.41 & 1.03 & 64.07\\
  	Architecture 3 & 11.58 & 179.15 & 12.72 & 266.55 & 12.04 & 215.09 & 0.35 & 28.39\\
  	Architecture 4 & 12.52 & 262.77 & 14.25 & 368.35 & 13.58 & 325.41 & 0.53 & 34.13\\
  	\hline
\end{tabular}
\caption{Results for different architectures for subset 1, 100 epochs}
\label{table:tested_architectures_100}
\end{table}

\begin{table}[!htb]
\centering
\begin{tabular}{l l l l}
	\hline
	Layer & Shape & Activation & Additional Information\\
  	\hline
  	Fully connected & 30 & ReLU & Dropout(0.6)\\
  	Fully connected & 10 & ReLU & Dropout(0.2)\\
  	Fully connected & 1 & Linear & \\
  	\hline
\end{tabular}
\caption{Proposed Neural Network architecture}
\label{table:proposed_nn}
\end{table}

\subsection{Shaping the data}

This section covers the data preprocessing applied to the raw sensor readings in each of the datasets. Even-though the original datasets contains $21$ different sensor readings some of the sensor do not present much variance and are therefore discarded. Therefore only $14$ sensor readings out of the $21$ are considered for this study, their indices are $\left\lbrace 2, 3, 4, 7, 8, 9, 11, 12, 13, 14, 15, 17, 20, 21 \right\rbrace$. The raw measurements are then used to create the strided time windows with window size $n_w$ and window stride $n_s$, for the labels $R_e$ is used at the early stages and then the \gls{rul} is linearly decreased. Finally, the data is normalized to be within the range $\left[ -1,1 \right]$ using the min-max normalization.

\begin{equation}
x^{i,j}_{norm} = 2 \frac{x^{i,j} - x^{j}_{min}}{x^{j}_{max} - x^{j}_{min}} - 1
\label{eq:min_max_norm}
\end{equation}

where $x_{i,j}$ denotes the original $i$-th data point of the $j$-th sensor and $x^{i,j_norm}$ is the normalized value of $x^{i,j}$.  $x^{j}_{max}$ and $x^{j}_{min}$ denote the maximum and minimum values of the original measurement data from the $j$-th sensor, respectively. 

\subsubsection{Time Window Processing}

In multivariate time-series based problems such as \gls{rul}, more information can be generally obtained from the temporal sequence of data as compared with the multivariate data point at a single time stamp. Let $n_w$ denote the size of the time window, for a time window with a stride $n_s = 1$, all the past sensor values within the time window are collected and put together to form a feature vector $\mathbf{x}$. This approach has successfully been tested in \cite{Li2018, Lim2016} where they propose the use of a moving window with values raging from 20 to 30. In this paper we propose not only the use of a moving time window, but also a \textit{strided} time window that updates $n_s$ elements at the time instead of $1$. 

The use of a \textit{strided time window} allows for the regressor to take advantage not only of the previous information available, but also to control the ratio at which the algorithm is fed with new information. With the usual time window approach only one point is updated for every new time window, on the contrary, the strided time window allows for updating $n_s$ points at the time, allowing for the algorithm to catch newer information with fewer iterations, furthermore, the information contained in the strided time window is likely more rich than the one contained in a time window with stride of $1$.

\subsubsection{Piecewise linear degradation model}

Different from common regression problems, the desired output value of the input data is difficult to determine for a \gls{rul} problem. It is usually impossible to evaluate the precise health condition and estimate the \gls{rul} of the system at each time step without an accurate physics based model. For this popular dataset, a piece-wise linear degradation model has been proposed in \cite{Ramasso2014}. The piece-wise linear degradation model assumes that the engines have a constant \gls{rul} label in the early cycles and then the \gls{rul} starts degrading linearly until it reaches 0. The piecewise linear degradation approach is used for this work, in here we denote the value for the \gls{rul} at the early stages as $R_e$. 

\subsection{Choosing optimal data-related parameters}

As mentioned in the previous sections the choice of the data-related parameters window size $n_w$, window stride $n_s$ and constant \gls{rul} $R_e$ have a large impact on the performance of the regressor, i.e. the \gls{mlp}. In this section we present a framework for picking the best combination of the data-related parameters $n_w$, $n_s$ and $R_e$ without spending too much computational time.

Let $\mathbf{v} = (n_w, n_s, R_e)$, where $n_w \in \left[1, b\right]$, $n_s \in \left[1, 10\right]$ and $R_e \in \left[90, 140 \right]$ and all the intervals are integer intervals. The value of $b$ is dependent upon the specific subset, Table \ref{table:b_values} presents the different values $b$ can take for each dataset.

\begin{table}[!htb]
\centering
\begin{tabular}{l | l l l l}
	\hline
	 & FD001 & FD002 2 & FD003 & FD004\\
  	\hline
  	$b$ & 30 & 20 & 30 & 18\\
  	\hline
\end{tabular}
\caption{Allowed values for $b$ per subset}
\label{table:b_values}
\end{table}

Given $\mathbf{v}$, we can evaluate the performance of the regressor by reshaping the data using $\mathbf{v}$, training the \gls{mlp} using the obtained data and then computing the scores in equations \ref{eq:rmse} and \ref{eq:rhs}.This setting is just what is required for performing any kind of optimization, i.e. to have a set of tunable parameters and a performance indicator, therefore, here we propose to fine tune $\mathbf{v}$ using a meta-heuristic algorithm.

\subsubsection{Differential Evolution for obtaining the optimal data-related parameters}

Differential Evolution (\gls{de}) \cite{Storn1997} is a method that optmizes a problem by iteratively trying to improve a candidate solution with regard to a given measure of quality. The method does not make any assumptions about the problem, therefore it is know as a metaheuristic, nevertheless, this kind of methods are not guaranteed to converge to the optimal solution. \gls{de} does not require the gradient of the problem being optimized, making it a very suitable metaheuristic for applications such as Neural Networks.

\gls{de} belongs to a class of algorithms known as evolutionary algorithms. The algorithm optimizes a problem by maintaining a \textit{population} of candidate solutions and creating new candidate solutions $\mathbf{v}'$ by combining existing ones according to a simple cross-over formula, keeping whichever candidate solution $\mathbf{v}^*$ has the best score or fitness on the optimization problem at hand. In this way the optimization problem is treated as a black box that merely provides a measure of quality given a candidate solution and the gradient is therefore not needed.

Although \gls{de} does not have special operators for treating integer variables a very simple modification to the algorithm, consisting on rounding every candidate solution $\mathbf{v}'$ to the nearest integer, is used for this work.

There is yet one last detail to be taken care of, evolutionary algorithms such as \gls{de} tend to use several function evaluations for obtaining the optimal solutions,  for our application one function evaluations implies retraining the Neural Network. Certainly this is not a desirable scenario, as obtaining the optimal parameter vector $\mathbf{v}$ would entail an extensive use of computational power. Instead of running for \gls{de} for several iterations and with a large population we propose to run it just for $10$ iterations (generations in the literature of evolutionary computation) and using a population size of $10$, which seems reasonable given the size of the search space of $\mathbf{v}$. Furthermore, during the tuning process the \gls{mlp} is not trained for  $100$ epochs, instead the \gls{mlp} is trained for just $20$ epochs, this is done mainly for two reasons: the use of the mini-batch in the training process allows for a speed up in the convergence, therefore it can be assumed that the algorithm will most likely be very close to its optima after just a couple of iterations, second and most important is the fact that we assume that parameters that lead to lower score values in the early stages of the \gls{mlp} training process are more likely to provide better performance overall. Details for the use of \gls{de} in finding the optimum data-related parameters are described in Table \ref{table:de_hyperparams}. The optimal data-related parameters for each of the subsets found by \gls{de} are shown in Table \ref{table:optimal_data_params}.

\begin{table}[!htb]
\centering
\begin{tabular}{l l l l l}
	\hline
	 Population Size & Generations & Strategy & \gls{mlp} epochs\\
  	\hline
  	10 & 10 & Best1Bin & 20\\
  	\hline
\end{tabular}
\caption{Differential Evolution hyper-parameters.}
\label{table:de_hyperparams}
\end{table}

\begin{table}[!htb]
\centering
\begin{tabular}{l l l l l}
	\hline
	 Dataset & Window Size $n_w$ & Window Stride $n_s$ & Early RUL $R_e$\\
  	\hline
  	FD001 & 26 & 2 & 100\\
  	FD002 & 16 & 2 & 91\\
  	FD003 & 30 & 2 & 97\\
  	FD004 & 16 & 2 & 92\\
  	\hline
\end{tabular}
\caption{Optimal data-related parameters for each subset.}
\label{table:optimal_data_params}
\end{table}
\section{Experimental Results}
\label{sec:results}

In this chapter we present the numerical results obtained with the \gls{eds} method on unconstrained, box constrained and mixed-integer problems up to three objectives. The \gls{eds} method is compared against the \gls{dzz} method \cite{zigzag_discrete} and \gls{nsga2} algorithm \cite{nsga2}. We chose to compare against the \gls{dzz} method since it is a state-of-the-art algorithm in the field of mixed-integer multi-objective optimization. As for \gls{nsga2}, it is one of the most widely used algorithms for solving \glspl{mop}. Although we are aware that there is no ``fair'' comparison between \gls{nsga2} and \gls{eds} algorithms given that the former is a \emph{global} optimization method while the latter is of \emph{local} nature, we consider this comparison necessary in order to show the quality of the solutions obtained by our method. 

\subsection{General Setting}

We will now describe the general setting of our experiments. \gls{dzz} version used in this experiment was provided by Dr. Honggang Wang from Rutgers University, being this the better version up to date. \gls{nsga2} algorithm was downloaded from \cite{ngpm_code} which is version of \gls{nsga2} with support for mixed-integer problems \cite{ngpm}. For each of the problems a similar number of solutions was computed by each algorithm in order to make a fairer comparison. For the case of \gls{nsga2} a budget of approximately four times the number of function evaluations spent by \gls{eds} was assigned as stopping criteria. In the case of the \gls{dzz} method the First Pareto Solution (\gls{fps}) is given in order to restrict the comparison to the continuation method (Zig and Zag steps). For each of the test problems the three algorithms were run ten times and the better solution of each of them is used for the comparison.

For each of the solved problems a table summarizing key information for the comparison is presented. We put special emphasis on the number of function evaluations used by each algorithm displayed in the \emph{Feval} column and on the value of the $\bigtriangleup_2$ and $\bigtriangleup_3$ performance indicators \cite{delta_p}. A ratio of function evaluations per solution is shown on the column \emph{Funeval/sol}.

Finally, for the computation of the $\bigtriangleup_p$ performance indicator the real Pareto front was used in all of the cases except for the last function presented in this section, where the Pareto front was approximated by performing ten \emph{long} runs of the \gls{nsga2} algorithm and keeping the non-dominated points among all of the runs.

\subsection{DTLZ2 Continuous Problem}

We now consider the tri-objective problem DTLZ1\cite{evolutionary_algorithms} which is a multimodal function with box constraints. It is defined as follows:

\begin{eqnarray}
& f_1(x) = & \frac{1}{2}x_1 x_2 (1 + g(x)) \nonumber \\
& f_2(x) = & \frac{1}{2}x_1 (1 - x_2) (1 + g(x)) \nonumber \\
& f_3(x) = & \frac{1}{2}(1 - x_1) (1 + g(x))\\
& g(x) = & 100(10 + \sum_{i = 3}^n(x_i - 0.5)^2 - cos(20 \pi (x_i - 0.5)))\nonumber \\
& s.t. & 0 \leq x_i \leq 1, \quad i = 1,\ldots,n. \nonumber
\end{eqnarray}

For this example we set $n = 3$. As in the previous example \gls{dzz} is not applicable to this function, hence the comparison is only made between the \gls{eds} method and \gls{nsga2} algorithm. Results are shown in Table \ref{table:results_dtlz1}.

\begin{table}[!htb]
\centering
\begin{tabular}{| l  r  r  r  r  r  r  r |}
	\hline
	Algorithm & Int. Var. & Real Var. & Sols & Feval & $\triangle_2$ & $\triangle_3$ & Funeval/Sol\\  
  	\hline
  	EDS & 0 & 3  & 870 & 5991  & 0.0085 & 0.0082 & 7\\
  	NSGA-II & 0 & 3 & 900 & 23400  & 0.0093 & 0.0094 & 26\\
  	\hline
\end{tabular}
\caption{Summarized results for DTLZ1 function}
\label{table:results_dtlz1}
\end{table}

Once again, as indicated by the results in Table \ref{table:results_dtlz1} the \gls{eds} method performs better than the \gls{nsga2} algorithm, despite the latter uses about four times more function evaluations. Figure \ref{Fig:dtlz1_function_pf} shows the Pareto fronts computed by both methods.

\begin{comment}

 \begin{figure}[H]
   \centering
	\hspace*{-0.5in}
    \subfloat[EDS]{%
      \includegraphics[width = 60mm, height = 60mm]{img/fronts/pf_dtlz1_eds.png}
    }
    \subfloat[NSGA-II]{%
      \includegraphics[width = 60mm, height = 60mm]{img/fronts/pf_dtlz1_ga.png}
    }
    \caption{Pareto fronts of the DTLZ2 function computed by the different methods}
    \label{Fig:dtlz1_function_pf}
\end{figure}

\end{comment}

\subsection{ZDT1 Integer Function}

Now we test on a discretized version of the ZDT1 function which proposed by Dr. Wang for his experiments with the \gls{dzz} method on \cite{zigzag_discrete}. We will use the same setting as the one proposed in \cite{zigzag_discrete}, the function is defined in the following way:

\begin{eqnarray}
& f_1(x) = & \frac{x_1}{100} \nonumber \\
& f_2(x) = & g(x) \left(1 - \sqrt{\frac{f_1(x)}{g(x)}} \right) \\
& g(x) = & 1 + 9 \cdot \frac{x_2 - x_1}{100}^2 \nonumber \\
& s.t. & 0 \leq x_i \leq 100, \quad i = 1, \ldots, n. \nonumber
\end{eqnarray}

where $n = 2$ and $x \in \mathbb{Z}^n$. This function has a convex Pareto front, the solution set is contained within the domain $[0,100] \times [0,100]$. The results for the three methods are displayed in Table \ref{table:results_zdt1Integer_function}.

\begin{table}[!htb]
\centering
\begin{tabular}{| l  r  r  r  r  r  r  r |}
	\hline
	Algorithm & Int. Var. & Real Var. & Sols & Feval & $\triangle_2$ & $\triangle_3$ & Funeval/Sol\\  
  	\hline
  	EDS & 0 & 2  & 122 & 268 & 0.0168 & 0.0324 & 2\\
  	NSGA-II & 0 & 2 & 120 & 1080  & 0.0165 & 0.0295 & 9\\
  	DZZ & 0 & 2 & 100 & 400 & 0.0100 & 0.0214 & 4\\
  	\hline
\end{tabular}
\caption{Summarized results for ZDT1 Integer function}
\label{table:results_zdt1Integer_function}
\end{table}

It can be observed by the results presented in Table \ref{table:results_zdt1Integer_function} that the three methods computed good solutions, being the $\bigtriangleup_p$ values of the three quite similar. Nevertheless, it can also be observed that the \gls{eds} method was the most efficient of the three methods, using up to four times less function evaluations than \gls{nsga2} and half the number of function evaluations of the \gls{dzz} method. The Pareto fronts obtained by each method are shown in Figure \ref{Fig:zdt1Integer_function_pf}.

\begin{comment}

\begin{figure}[H]
   \centering
	\hspace*{-0.5in}
    \subfloat[EDS]{%
      \includegraphics[width = 50mm, height = 50mm]{img/fronts/pf_ijocCase1_eds.png}
    }
    \subfloat[NSGA-II]{%
      \includegraphics[width = 50mm, height = 50mm]{img/fronts/pf_ijocCase1_ga.png}
    }
    \subfloat[DZZ]{%
      \includegraphics[width = 50mm, height = 50mm]{img/fronts/pf_ijocCase1_dzz.png}
    }
    \caption{Pareto fronts of the ZDT1 integer function computed by the different methods}
    \label{Fig:zdt1Integer_function_pf}
\end{figure}

\end{comment}

\subsection{ZDT2 Integer Function}

Our next test is conducted on a discretized version of the ZDT2 function proposed in \cite{zigzag_discrete}. Once again we will use the same setting as the one proposed in \cite{zigzag_discrete}. The function is defined in the as follows:

\begin{eqnarray}
& f_1(x) = & \frac{x_1}{100} \nonumber \\
& f_2(x) = & g(x) \cdot \left(1 - \sqrt{\frac{f_1(x)}{g(x)}} \right)^2 \\
& g(x) = & 1 + \left(\frac{x_2 }{100} \right)^\frac{1}{4} \nonumber \\
& s.t. & 0 \leq x_i \leq 100, \quad i = 1, \ldots, n. \nonumber
\end{eqnarray}

where $n = 2$ and $x \in \mathbb{Z}^n$. This function has a concave Pareto front, the solution set is contained within the square $[0,100] \times [0,100]$. The results for the three methods are displayed in Table \ref{table:results_zdt2Integer_function}.

\begin{table}[!htb]
\centering
\begin{tabular}{| l  r  r  r  r  r  r  r |}
	\hline
	Algorithm & Int. Var. & Real Var. & Sols & Feval & $\triangle_2$ & $\triangle_3$ & Funeval/Sol\\  
  	\hline
  	EDS & 0 & 2  & 24 & 49 & 0.044 & 0.065 & 2\\
  	NSGA-II & 0 & 2 & 20 & 200  & 0.15 & 0.21 & 10\\
  	DZZ & 0 & 2 & 26 & 106 & 0.031 & 0.051 & 4\\
  	\hline
\end{tabular}
\caption{Summarized results for ZDT2 Integer function}
\label{table:results_zdt2Integer_function}
\end{table}

The results presented in Table \ref{table:results_zdt2Integer_function} demonstrate, once again, that the results obtained by the \gls{eds} and the \gls{dzz} methods are similar with respect to the $\bigtriangleup_p$ indicator. Nevertheless, the \gls{eds} method uses half of the function evaluations that the \gls{dzz} method uses. \gls{nsga2} is again outperformed by both methods. The Pareto fronts obtained by each method are shown in Figure \ref{Fig:zdt2Integer_function_pf}.

\begin{comment}

\begin{figure}[H]
	\centering
	\hspace*{-0.5in}
    \subfloat[EDS]{%
      \includegraphics[width = 50mm, height = 50mm]{img/fronts/pf_ijocCase2_eds.png}
    }
    \subfloat[NSGA-II]{%
      \includegraphics[width = 50mm, height = 50mm]{img/fronts/pf_ijocCase2_ga.png}
    }
    \subfloat[DZZ]{%
      \includegraphics[width = 50mm, height = 50mm]{img/fronts/pf_ijocCase2_dzz.png}
    }
    \caption{Pareto fronts of the ZDT2 Integer function computed by the different methods}
    \label{Fig:zdt2Integer_function_pf}
\end{figure}

\end{comment}

\subsection{A Mixed-Integer Model}

Finally, in this section we present one mixed-integer problem solved by the \gls{eds} method. A comparison between \gls{eds} and \gls{nsga2} algorithms is presented. No comparison against the \gls{dzz} method is possible since this is a tri-objective. Its definition is as follows:

\begin{eqnarray}
f_1(x) = || x - a_1 ||_2^2 \nonumber \\
f_2(x) = || x - a_2 ||_2^2 \\
f_3(x) = || x - a_3 ||_2^2,\nonumber
\label{eq:f3d_mip_fun}
\end{eqnarray}

for $a_1 = (20, \ldots, 20) \in \mathbb{R}^d$, $a_2 = -a_1$ and $a_3 = (\underbrace{20, \ldots, 20}_\text{m times}, \underbrace{-20, \ldots, -20}_\text{n-m times}) \in \mathbb{R}^d$ for $m = ceil(d/2)$, $d = d_1 + d_2$, $d_1 = 3$, $d_2 = 2$, and $x \in \mathbb{R}^{d_1} \times \mathbb{Z}^{d_2}$. This function has a convex Pareto front. Since comparison against \gls{dzz} is possible only the comparison against \gls{nsga2} is presented. The results of such comparison are shown in Table \ref{table:results_binh3mip}.

\begin{table}[!htb]
\centering
\begin{tabular}{| l  r  r  r  r  r  r  r |}
	\hline
	Algorithm & Int. Var. & Real Var. & Sols & Feval & $\triangle_2$ & $\triangle_3$ & Funeval/Sol\\  
  	\hline
  	EDS & 2 & 3  & 434 & 6995  & 135.88 & 150.74 & 9\\
  	NSGA-II & 2 & 3 & 450 & 27000  & 136.69 & 145.36 & 38\\
  	\hline
\end{tabular}
\caption{Summarized results for Binh3 MI function}
\label{table:results_binh3mip}
\end{table}

As the data in Table \ref{table:results_binh3mip} shows, both methods, the \gls{eds} method and the \gls{nsga2} algorithm deliver similar quality solutions. Nevertheless it is important to note that the \gls{eds} method used four times less function evaluations than \gls{nsga2} to reach the same quality of solutions. The high values on the $\bigtriangleup_p$ values of each method are due to the scale of the objective space, going up to 8000 units in one of the objective functions. The fronts computed by both methods are shown in Figure \ref{Fig:binh3mip_function_pf}.

\begin{comment}

 \begin{figure}[H]
 	\centering
	\hspace*{-0.5in}
    \subfloat[EDS]{%
      \includegraphics[width = 60mm, height = 60mm]{img/fronts/pf_f3dmip_eds.png}
    }
    \subfloat[NSGA-II]{%
      \includegraphics[width = 60mm, height = 60mm]{img/fronts/pf_f3dmip_ga.png}
    }
    \caption{Pareto fronts of the Binh3 MI function computed by the different methods}
    \label{Fig:binh3mip_function_pf}
\end{figure}

\end{comment}

It can be observed in the above pictures that the \gls{eds} method had some difficulties computing the narrow parts of the Pareto front, nevertheless, it can also be observed that the solutions of the \gls{eds} method are more evenly distributed than those computed by the \gls{nsga2}. This explains why the $\bigtriangleup_p$ values of both methods are similar. The higher values of the $\bigtriangleup_3$ indicator on the \gls{eds} method are due to the fact that as $p$ increases in the $p$-norm, outliers are penalized more (see \cite{delta_p}).
\section{Conclusions and future work}
\label{sec:conclusions}

This paper presents a novel framework for predicting the \gls{rul} of mechanical components. While the method was tested on the jet-engine specific dataset \gls{cmaps}, the method is general enough so that it can theoretically be applied to other kind of similar systems. The framework makes use of a strided moving time window to generate the training and test sets, a shallow \gls{mlp} to make the predictions of the \gls{rul} and an evolutionary algorithm (\gls{de}) which needs to be run just once in order to find the best data-related parameters that optimize the scoring functions used in this study.  The results presented in this paper demonstrate that the proposed framework is accurate and computationally efficient, which makes this framework suitable for applications that have limited computational resources such as embedded systems. Furthermore, a comparison with other state-of-the-art methods shown that the proposed method is the best overall performer. 

Two major features of the proposed framework are its generality and scalability. While for this paper very specific regressors and evolutionary algorithms were chosen, many other combinations are possible and may be more suitable for different applications. Furthermore, the framework here presented can, in principle, be used for model-construction, i.e. generating the best possible neural network architecture tailored to a specific application. Both issues are to be addressed in future work.

%------------------------------------------------


%----------------------------------------------------------------------------------------
%	REFERENCE LIST
%----------------------------------------------------------------------------------------

\addcontentsline{toc}{section}{References}
\bibliographystyle{unsrt}
\bibliography{reference_rul_paper}

%----------------------------------------------------------------------------------------

\end{document}
